%%%%%%%%%%%%%%%%%%%%%%%%%%%%%%%%%%%%%%%%%
% Medium Length Professional CV
% LaTeX Template
% Version 2.0 (8/5/13)
%
% This template has been downloaded from:
% http://www.LaTeXTemplates.com
%
% Original author:
% Thanks : Rishi Shah 's Contribution
% inspired by his awesome contribution:
% https://www.overleaf.com/articles/rishi-shahs-resume/vgxvkmxktyxn
% Author : Allianzcortex
% contact me : github.com/Allianzcortex
% email : iamwanghz#gmail.com
%
% Important note:
% This template requires the resume.cls file to be in the same directory as the
% .tex file. The resume.cls file provides the resume style used for structuring the
% document.
%
%%%%%%%%%%%%%%%%%%%%%%%%%%%%%%%%%%%%%%%%%

%----------------------------------------------------------------------------------------
%	PACKAGES AND OTHER DOCUMENT CONFIGURATIONS
%----------------------------------------------------------------------------------------

\documentclass{resume} % Use the custom resume.cls style
\usepackage{blindtext}

\usepackage{hyperref}

\hypersetup{
    colorlinks=true,
    linkcolor=blue,
    filecolor=magenta,      
    urlcolor=cyan,
    pdftitle={Overleaf Example},
    pdfpagemode=FullScreen,
    }
    
\urlstyle{same}


\usepackage[left=0.40in,top=0.4in,right=0.40in,bottom=0.4in]{geometry} % Document margins
\usepackage{fontawesome}
\usepackage{times}
\newcommand{\tab}[1]{\hspace{.2667\textwidth}\rlap{#1}}
\newcommand{\itab}[1]{\hspace{0em}\rlap{#1}}
% \begin{center}
% {\centerline {\em \textbf {Seeking for a fulltime internship from Sep 2019 - Apr 2010(8 months) } } }
% \end{center}
\name{Zeyong Jin} % Your name 



\address{\faPhone \, (778) 316-8193 \: \: \faEnvelope{ zeyongj@gmail.com} \: \: \faGithub{ \url{https://github.com/zeyongj}} \: \: \faLink{ \url{https://www.zeyongjin.net/}}}
\address{\faMapMarker{ 435 Braid Street, New Westminster, BC, V3L 5M5, Canada}} 

\begin{document}

%----------------------------------------------------------------------------------------
%	EDUCATION SECTION
%----------------------------------------------------------------------------------------

\begin{rSection}{Education}

{\bf Simon Fraser University} \hfill {\em Sep 2018 - Apr 2022} 
\\{ \textit {Bachelor of Science (with Distinction)}} \hfill {Burnaby, BC} 
\begin{itemize}
  \vspace{-0.2cm}\item \textbf{Program}: Major in Computing Science, Concentration in Information Systems, Minor in Mathematics.
  \vspace{-0.2cm}\item \textbf{Grade}: Cumulative GPA of 3.62.
  \vspace{-0.2cm}\item \textbf{Related Courses}: Data Structures, Algorithms, Computer Systems, Operating Systems, Software Engineering, Requirements Engineering, Database Systems, Computer Graphics, UI Design, Natural Language Processing, Computational Data Science, Data Mining, Linear Optimization, Number Theory, Graph Theory, Abstract Algebra.
\end{itemize}

{\bf Fraser International College} \hfill {\em May 2017 - Apr 2018} 
\\{ \textit {UTP Stage II:  Science}} \hfill {Burnaby, BC} 
\begin{itemize}
  \vspace{-0.2cm}\item \textbf{Program}: Major in Computing Science.
  \vspace{-0.2cm}\item \textbf{Grade}: Cumulative GPA of 3.92.
  \vspace{-0.2cm}\item \textbf{Award}: Dean’s Honour Roll \textit{(2017 Summer)}.
  \vspace{-0.2cm}\item \textbf{Related Courses}: C++ Programming, Calculus, Discrete Mathematics, Linear Algebra.
\end{itemize}


\end{rSection}

\begin{rSection}{Technical Skills}
\begin{itemize}
    \item \textbf{Programming Languages}:  C, C++, Java, Python, R, Assembly, C\#.
    \vspace{-0.2cm}\item \textbf{IDEs \& Development Tools}: Microsoft Visual Studio, IntelliJ IDEA, Android Studio, PyCharm, R Studio, Google Colab, Jupyter Notebook.
    \vspace{-0.2cm}\item \textbf{Database Systems}: Microsoft SQL Server 2018, MySQL.
    \vspace{-0.2cm}\item \textbf{Operating Systems}: Linux \textit{(Ubuntu)}, Windows.
    \vspace{-0.2cm}\item \textbf{Version Control Tools}: Git \textit{(GitLab, GitHub)}.
    \vspace{-0.2cm}\item \textbf{UI/UX Design Tools}: Balsamiq, Figma.
    \vspace{-0.2cm}\item \textbf{Documentation \& Office Tools}: \LaTeX, R Markdown, Microsoft Office Suite \textit{(Word, Excel, PowerPoint)}.
    \vspace{-0.2cm}\item \textbf{Certificates}: Machine Learning Specification \textit{(Provided by Stanford University on Coursera)}; Introduction to Data Science in Python, Applied Machine Learning in Python \textit{(Provided by the University of Michigan on Coursera)}.
\end{itemize}

\end{rSection}

\begin{rSection}{Transferable Skills and Interests}
\begin{itemize}
    \item \textbf{Execution Ability}: Proficient in translating strategies and ideas into actionable plans and executing them.
    \vspace{-0.2cm} \item \textbf{Communication \& Teamwork}: Excelled in leading diverse teams, fostering collaboration and persuading peers.
    \vspace{-0.2cm} \item \textbf{Time Management}: Consistently meets deadlines through effective prioritization and planning.
    \vspace{-0.2cm} \item \textbf{Continuous Learning}: Possesses a strong curiosity and openness to acquiring new knowledge and skills.
    \vspace{-0.2cm} \item \textbf{Languages}: Bilingual in Mandarin \textit{(Native)} and English \textit{(Proficient)}.
    % \vspace{-0.2cm} \item \textbf{Interests}: Enjoys reading, writing, cultural immersion and historical site exploration in leisure time.
\end{itemize}
\end{rSection}


\begin{rSection}{Professional EXPERIENCES}

{\bf Rancho Management Services (B.C.) Ltd.} \hfill {\em Apr 2024 - Present} 
\\{\textit{Junior Accounts Payable} \hfill {Vancouver, BC}}
\begin{itemize}
    \vspace{-0.2cm}\item Managing and processing accounts and incoming payments, ensuring adherence to financial policies and procedures, thereby maintaining system integrity and data accuracy.
    \vspace{-0.2cm}\item Performing daily financial transactions, including verifying, classifying, and recording accounts payable data, enhancing proficiency in data management and analytics.
    \vspace{-0.2cm}\item Preparing bills, invoices, and bank deposits, developing meticulous documentation and precise data entry skills.
\end{itemize}

{\bf Simon Fraser University} \hfill {\em Jan 2021 - Dec 2021} \\
{\textit{Teaching Assistant} \hfill {Burnaby, BC}}
\begin{itemize}
    \vspace{-0.2cm}\item Facilitated grading processes using Crowdmark for the course STAT 203 \textit{(Introduction to Statistics for the Social Sciences)}, providing prompt and constructive feedback to students and instructors \textit{(Spring Term)}.
    \vspace{-0.2cm}\item Assisted in grading assignments and exams via Canvas for the course CMPT 115 \textit{(Exploring Computer Science)}, ensuring timely feedback to both students and faculty \textit{(Fall Term)}.
    \vspace{-0.2cm}\item Hosted weekly online office hours, addressing undergraduate queries related to computing science. Addressed student inquiries regarding grading, ensuring all responsibilities were completed punctually.
    \vspace{-0.2cm}\item Contributed to the Department of Statistics and Actuarial Science, working alongside Dr. Harsha Perera and Mr. Scott Pai. Supported the School of Computing Science under the supervision of Dr. Diana Cukierman.
\end{itemize}



\end{rSection}

\begin{rSection}{Academic Projects}

% {\bf Prompt-based Text Matching Methods for Fake News Stance Detection} \hfill {\em Sep 2021 - Dec 2021} 
% \\{\textit{CMPT 413/713: Computational Linguistics (Natural Language Processing)} \hfill {Burnaby, BC}}
% \begin{itemize}
%     \vspace{-0.2cm}\item Used the Bidirectional Encoder Representations from Transformers \textit{(BERT)} model to process news headlines and news article body and based on fine-tuning the model to continue training the BERT model for detecting the stance relationship between news headlines and news article body. 
%     \vspace{-0.2cm}\item Implemented a fake news classification model based on BERT and compared the results of the BERT model with four baseline algorithms. 
%     \vspace{-0.2cm}\item Processed Fake New Competition \textit{(FNC)} task data via BERT, and obtained an accuracy of 90.37\%.
%     \vspace{-0.2cm}\item Coded in Python via the platform of Google Colab and instructed by Prof. Angel Chang. Project available at \url{https://github.com/zeyongj/Fake-News-Stance-Detection}, presentation available at \url{https://www.youtube.com/watch?v=DtKjSMv31RQ&ab_channel=ZeyongJin}.
% \end{itemize}

{\bf Prompt-based Text Matching Methods for Fake News Stance Detection} \hfill {\em Sep 2021 - Dec 2021} 
\\{\textit{CMPT 413/713: Computational Linguistics (Natural Language Processing)} \hfill {Burnaby, BC}}
\begin{itemize}
    \vspace{-0.2cm} \item Leveraged the Bidirectional Encoder Representations from Transformers \textit{(BERT)} model to analyze the stance relationship between news headlines and their corresponding articles.
    \vspace{-0.2cm} \item Fine-tuned the BERT model for enhanced accuracy in detecting stance relationships, achieving a notable accuracy of 90.37\% on the Fake News Competition \textit{(FNC)} dataset.
    \vspace{-0.2cm} \item Benchmarked the BERT-based model against four baseline algorithms, highlighting its superior performance in fake news classification.
    \vspace{-0.2cm} \item Developed the entire solution in Python using Google Colab under the guidance of Prof. Angel Chang.
    \vspace{-0.2cm} \item Project repository and detailed documentation available at \url{https://github.com/zeyongj/Fake-News-Stance-Detection}. A comprehensive presentation can be viewed at \url{https://www.youtube.com/watch?v=DtKjSMv31RQ&ab_channel=ZeyongJin}.
\end{itemize}



{\bf COVID-19 Patient Outcome Prediction} \hfill {\em Jan 2021 - Apr 2021} 
\\{\textit{CMPT 459: Data Mining} \hfill {Burnaby, BC}}
\begin{itemize}
    \vspace{-0.2cm}\item Engineered a predictive model to determine potential outcomes, namely recovered, hospitalized and non-hospitalized for COVID-19 patients.
    \vspace{-0.2cm}\item Presented data visualizations, including a global heatmap of confirmed cases and comprehensive attribute statistics from datasets.
    \vspace{-0.2cm}\item Built, evaluated, and proceeded parameter tunning on classifiers of Light Gradient Boosting Machine \textit{LightGBM}, Support Vector Machine \textit{(SVM)} and Multilayer Perceptron \textit{(MLP)}. Achieved highest accuracy of 0.8840 and F1 score of 0.8391 with LightGBM.
    \vspace{-0.2cm}\item Coded in Python via the platform of Google Colab and instructed by Prof. Martin Ester. Project available at \url{https://github.com/zeyongj/Prediction-of-The-Outcome-of-A-COVID-19-Patient}.
\end{itemize}

{\bf Practical Parent Application} \hfill {\em Sep 2020 - Dec 2020} 
\\{\textit{CMPT 276: Introduction to Software Engineering} \hfill {Burnaby, BC}}
\begin{itemize}
    \vspace{-0.2cm}\item Developed an Android application in Android Studio to assist parents in task delegation among children, incorporating features like a coin-flip game for fair task assignment.
    \vspace{-0.2cm}\item Integrated real-time task status tracking and a unique breathing rhythm regulator to guide parents in structured breathing exercises, enhancing user experience and utility.
    \vspace{-0.2cm}\item Collaborated in a team of four, leveraging agile methodologies for iterative development, and produced comprehensive user stories and development documentation.
    \vspace{-0.2cm}\item Project mentored by Dr. Brian Fraser, with detailed documentation and code available at \url{https://github.com/zeyongj/Pratical-Parent-Application}.
\end{itemize}


% {\bf Improving Dijkstra's Algorithm} \hfill {\em Sep 2019 - Dec 2019}
% \\{\textit{CMPT 307: Data Structures and Algorithms} \hfill {Burnaby, BC}}
% \begin{itemize}
%     \vspace{-0.2cm}\item Found the shortest distance and path of any two places in a city using Dijkstra’s, A* and Landmark Algorithms and showed the savings if A* and Landmark Algorithms are used.
%     \vspace{-0.2cm}\item Exported all the results as a text file.
%     \vspace{-0.2cm}\item Coded in C++ via the integrated development environment of Microsoft Visual Studio 2017, and instructed by Prof. Binay Bhattacharya. Project available at \url{https://github.com/zeyongj/Improving-Dijkstra-Algorithm}.
% \end{itemize}

\end{rSection}

\begin{rSection}{Personal Projects}

{\bf Advanced House Price Prediction} \hfill {\em Jan 2023 - Apr 2023}
\\{\textit{Kaggle Project} \hfill {New Westminster, BC}}
\begin{itemize}
    \vspace{-0.2cm}\item Developed an ensemble machine learning model to predict house prices using the Ames Housing dataset. The model employed a stacking technique with RandomForestRegressor, XGBRegressor, and Lasso as base models and a final estimator Lasso model.
    \vspace{-0.2cm}\item Implemented an effective preprocessing pipeline to handle both numerical and categorical data. Conducted comprehensive exploratory data analysis to discover underlying trends and patterns, informing feature engineering and selection strategies.
    \vspace{-0.2cm}\item Leveraged GridSearchCV for hyperparameter tuning and KFold cross-validation for model evaluation. Achieved a competitive mean RMSE score, identifying opportunities for further model enhancement, such as extensive feature selection and advanced ensemble methods.
    \vspace{-0.2cm}\item Coded in Python. Project available at \url{https://github.com/zeyongj/House-Prices-Advanced-Regression-Techniques}.
\end{itemize}

% {\bf Countdown Timer Application} \hfill {\em Sep 2020 - Oct 2020} 
% \\{\textit{Android Development Project} \hfill {New Westminster, BC}}
% \begin{itemize}
%     \vspace{-0.2cm}\item Crafted an intuitive Android application enabling users to set customizable countdown timers, with real-time display of remaining time.
%     \vspace{-0.2cm}\item Implemented an alert system to notify users with an alarm upon timer completion, enhancing user experience.
%     \vspace{-0.2cm}\item Developed in Java using Android Studio. Code at \url{https://github.com/zeyongj/Count-Down-Timer-Android-App}.
% \end{itemize}


\end{rSection}

\end{document}
