%%%%%%%%%%%%%%%%%%%%%%%%%%%%%%%%%%%%%%%%%
% Medium Length Professional CV
% LaTeX Template
% Version 2.0 (8/5/13)
%
% This template has been downloaded from:
% http://www.LaTeXTemplates.com
%
% Original author:
% Thanks : Rishi Shah 's Contribution
% inspired by his awesome contribution:
% https://www.overleaf.com/articles/rishi-shahs-resume/vgxvkmxktyxn
% Author : Allianzcortex
% contact me : github.com/Allianzcortex
% email : iamwanghz#gmail.com
%
% Important note:
% This template requires the resume.cls file to be in the same directory as the
% .tex file. The resume.cls file provides the resume style used for structuring the
% document.
%
%%%%%%%%%%%%%%%%%%%%%%%%%%%%%%%%%%%%%%%%%

%----------------------------------------------------------------------------------------
%	PACKAGES AND OTHER DOCUMENT CONFIGURATIONS
%----------------------------------------------------------------------------------------

\documentclass{resume} % Use the custom resume.cls style
\usepackage{blindtext}

\usepackage{hyperref}

\hypersetup{
    colorlinks=true,
    linkcolor=blue,
    filecolor=magenta,      
    urlcolor=cyan,
    pdftitle={Overleaf Example},
    pdfpagemode=FullScreen,
    }
    
\urlstyle{same}


\usepackage[left=0.40in,top=0.4in,right=0.40in,bottom=0.4in]{geometry} % Document margins
\usepackage{fontawesome}
\usepackage{times}
\newcommand{\tab}[1]{\hspace{.2667\textwidth}\rlap{#1}}
\newcommand{\itab}[1]{\hspace{0em}\rlap{#1}}
% \begin{center}
% {\centerline {\em \textbf {Seeking for a fulltime internship from Sep 2019 - Apr 2010(8 months) } } }
% \end{center}
\name{Zeyong Jin} % Your name 



\address{\faPhone \, (778) 316-8193 \: \: \faEnvelope{ zeyongj@gmail.com} \: \: \faGithub{ \url{https://github.com/zeyongj}} \: \: \faLink{ \url{https://www.zeyongjin.net/}}}
\address{\faMapMarker{ 435 Braid Street, New Westminster, BC, V3L 5M5, Canada}} 

\begin{document}

%----------------------------------------------------------------------------------------
%	EDUCATION SECTION
%----------------------------------------------------------------------------------------

\begin{rSection}{Education}

{\bf Simon Fraser University} \hfill {\em Sep 2018 - Apr 2022} 
\\{ \textit {Bachelor of Science (with Distinction)}} \hfill {Burnaby, BC} 
\begin{itemize}
  \vspace{-0.2cm}\item \textbf{Program}: Major in Computing Science, Concentration in Information Systems, Minor in Mathematics.
  \vspace{-0.2cm}\item \textbf{Grade}: Cumulative GPA of 3.62.
  \vspace{-0.2cm}\item \textbf{Related Courses}: Data Structures, Algorithms, Computer Systems, Operating Systems, Software Engineering, Requirements Engineering, Database Systems, Computer Graphics, UI Design, Natural Language Processing, Computational Data Science, Data Mining, Linear Optimization, Number Theory, Graph Theory, Abstract Algebra.
\end{itemize}

{\bf Fraser International College} \hfill {\em May 2017 - Apr 2018} 
\\{ \textit {UTP Stage II:  Science}} \hfill {Burnaby, BC} 
\begin{itemize}
  \vspace{-0.2cm}\item \textbf{Program}: Major in Computing Science.
  \vspace{-0.2cm}\item \textbf{Grade}: Cumulative GPA of 3.92.
  \vspace{-0.2cm}\item \textbf{Award}: Dean’s Honour Roll \textit{(2017 Summer)}.
  \vspace{-0.2cm}\item \textbf{Related Courses}: C++ Programming, Calculus, Discrete Mathematics, Linear Algebra.
\end{itemize}


\end{rSection}

\begin{rSection}{Technical Skills}
\begin{itemize}
    \item \textbf{Programming Languages}:  C, C++, Java, Python, MATLAB, Assembly, R.
    \vspace{-0.2cm}\item \textbf{Development Tools}:  Microsoft Visual Studio, IntelliJ IDEA, Android Studio, PyCharm, R Studio.
    \vspace{-0.2cm}\item \textbf{Database Systems}: Microsoft SQL Server 2018, MySQL.
    \vspace{-0.2cm}\item \textbf{Operating Systems}: Linux \textit{(Ubuntu)}, Windows.
    \vspace{-0.2cm}\item \textbf{Version Control Tool}: Git \textit{(GitLab, GitHub)}.
    \vspace{-0.2cm}\item \textbf{User Interface Design Tools}: Balsamiq, Figma.
    \vspace{-0.2cm}\item \textbf{Others}: \LaTeX, R Markdown, Microsoft Office Suit.
\end{itemize}
\end{rSection}

\begin{rSection}{Transferable Skills and Interests}
\begin{itemize}
    \item \textbf{Strong Execution Ability}: Able to translate strategy and ideas into execution.
    \vspace{-0.2cm}\item \textbf{Communication and Teamwork Skills}:  Convince students from different countries in a group.
    \vspace{-0.2cm}\item \textbf{Time Management Ability}: Finish assigned work on time via making prioritization and planning.
    \vspace{-0.2cm}\item \textbf{Endless Curiosity}: Welcome and willing to learn new knowledge and develop new skills.
    \vspace{-0.2cm}\item \textbf{Languages}: Mandarin \textit{(Native)}, English \textit{(Proficient)}.
    \vspace{-0.2cm}\item \textbf{Interests}: Reading, Writing, Coding.
\end{itemize}
\end{rSection}

\begin{rSection}{Professional EXPERIENCE}

{\bf Simon Fraser University} \hfill {\em Sep 2021 - Dec 2021} 
\\{\textit{Teaching Assistant, School of Computing Science} \hfill {Burnaby, BC}}
\begin{itemize}
    \vspace{-0.2cm}\item Helped instructors mark homework and exams using the platform of Canvas and provided feedback to both instructors and students on time. 
    \vspace{-0.2cm}\item Scheduled online office hour per week to assist undergraduates with computing science-related questions.
    \vspace{-0.2cm}\item Course Conducted: CMPT 115 \textit{(Exploring Computer Science)}, supervised by Dr. Diana Cukierman.
\end{itemize}

{\bf Simon Fraser University} \hfill {\em Jan 2021 - Apr 2021} 
\\{\textit{Teaching Assistant, Department of Statistics and Actuarial Science} \hfill {Burnaby, BC}}
\begin{itemize}
    \vspace{-0.2cm}\item Helped instructors mark homework and exams using the platform of Crowdmark and provided feedback to both instructors and students on time. 
    \vspace{-0.2cm}\item Answered students’ questions about grading  and finished the assigned work on time.
    \vspace{-0.2cm}\item Course conducted: STAT 203 \textit{(Introduction to Statistics for the Social Sciences)}, supervised by Dr. Harsha Perera and Mr. Scott Pai.
\end{itemize}


\end{rSection}

\begin{rSection}{Academic Projects}

{\bf Prompt-based Text Matching Methods for Fake News Stance Detection} \hfill {\em Sep 2021 - Dec 2021} 
\\{\textit{CMPT 413/713: Computational Linguistics (Natural Language Processing)} \hfill {Burnaby, BC}}
\begin{itemize}
    \vspace{-0.2cm}\item Used the Bidirectional Encoder Representations from Transformers \textit{(BERT)} model to process news headlines and news article body and based on fine-tuning the model to continue training the BERT model for detecting the stance relationship between news headlines and news article body. 
    \vspace{-0.2cm}\item Implemented a fake news classification model based on BERT and compared the results of the BERT model with four baseline algorithms. 
    \vspace{-0.2cm}\item Processed Fake New Competition \textit{(FNC)} task data via BERT, and obtained an accuracy of 90.37\%.
    \vspace{-0.2cm}\item Coded in Python via the platform of Google Colab and instructed by Prof. Angel Chang. Project available at \url{https://github.com/zeyongj/Fake-News-Stance-Detection}, presentation available at \url{https://www.youtube.com/watch?v=DtKjSMv31RQ&ab_channel=ZeyongJin}.
\end{itemize}

{\bf COVID-19 Patient Outcome Prediction} \hfill {\em Jan 2021 - Apr 2021} 
\\{\textit{CMPT 459: Data Mining} \hfill {Burnaby, BC}}
\begin{itemize}
    \vspace{-0.2cm}\item Developed a program to predict the most possible outcome, namely recovered, hospitalized and non-hospitalized of a COVID-19 patient.
    \vspace{-0.2cm}\item Showed visualizations, such as a heat map of confirmed cases worldwide and statistics for all attributes recorded in datasets. 
    \vspace{-0.2cm}\item Built, evaluated, and proceeded parameter tunning on classifiers of Light Gradient Boosting Machine \textit{LightGBM}, Support Vector Machine \textit{(SVM)} and Multilayer Perceptron \textit{(MLP)}, and acknowledged that LightGBM model has the best prediction accuracy of 0.8840 and F1 score of 0.8391. In contrast, the SVM model has the worst prediction accuracy of 0.8178 and worst F1 score of 0.7436.
    \vspace{-0.2cm}\item Coded in Python via the platform of Google Colab and instructed by Prof. Martin Ester. Project available at \url{https://github.com/zeyongj/Prediction-of-The-Outcome-of-A-COVID-19-Patient}.
\end{itemize}

{\bf Practical Parent Application} \hfill {\em Sep 2020 - Dec 2020} 
\\{\textit{CMPT 276: Introduction to Software Engineering} \hfill {Burnaby, BC}}
\begin{itemize}
    \vspace{-0.2cm}\item Developed an app to help parents to determine the order of their children to do a certain task, using Android Studio as the integrated development environment to implement features. 
    \vspace{-0.2cm}\item Created a feature for parents to use a game of flipping a coin to decide the turn of children for a certain task and check the status of each task.
    \vspace{-0.2cm}\item Provided a feature of regulating breathing rhythm for parents, helping parents to do standard ``hale in” and ``breath out” for certain times. 
    \vspace{-0.2cm}\item Prepared documents, including user stories and development documentation acting as the user manual, and instructed by Dr. Brian Fraser. Project available at \url{https://github.com/zeyongj/Pratical-Parent-Application}.
\end{itemize}

{\bf Improving Dijkstra's Algorithm} \hfill {\em Sep 2019 - Dec 2019}
\\{\textit{CMPT 307: Data Structures and Algorithms} \hfill {Burnaby, BC}}
\begin{itemize}
    \vspace{-0.2cm}\item Found the shortest distance and path of any two places in a city using Dijkstra’s, A* and Landmark Algorithms and showed the savings if A* and Landmark Algorithms are used.
    \vspace{-0.2cm}\item Exported all the results as a text file.
    \vspace{-0.2cm}\item Coded in C++ via the integrated development environment of Microsoft Visual Studio 2017, and instructed by Prof. Binay Bhattacharya. Project available at \url{https://github.com/zeyongj/Improving-Dijkstra-Algorithm}.
\end{itemize}

\end{rSection}

\begin{rSection}{Personal Projects}

{\bf Countdown Timer Application} \hfill {\em Sep 2020 - Oct 2020} 
\begin{itemize}
    \vspace{-0.2cm}\item Created an Android application to make users set up a countdown timer as they want, and the user could directly see how much time is left by the number on the screen.
    \vspace{-0.2cm}\item Managed the timer, so when it comes to an end, the user will receive a notification with an alarm.
    \vspace{-0.2cm}\item Coded in Java, using Android Studio as the integrated development environment. Code available at \url{https://github.com/zeyongj/Count-Down-Timer-Android-App}.
\end{itemize}

\end{rSection}

\end{document}
